\documentclass{article}
\usepackage[utf8]{inputenc}
\usepackage{indentfirst}
%\usepackage{mathptmx}%Times
\title{CS131 B1: HW 2}
\author{Duy Nguyen}
\date{24 September 2016}

\begin{document}

\maketitle

\section*{Question 1}

\subsection*{a. Pluto is called a planet if and only if Kuiper Belt objects are called planets.}

Pluto is called a planet \textbf{P}

Kuiper Belt objects are called planet \textbf{K}

$P \leftrightarrow K$

P is false, K is false so $P \leftrightarrow K$ is true.

\subsection*{b. If C++ is object-oriented language, then Java is also an object-oriented language.}

C++ is object-oriented language \textbf{C}

Java is an object-oriented language \textbf{J}

$C \rightarrow J$

C is true and J is true so $C \rightarrow J$ is true.

\subsection*{c. If C++ uses pointers, then Java also uses pointers.}

C++ uses pointers \textbf{C}

Java uses pointers \textbf{J}

$C \rightarrow J$

C is true and J is false so $C \rightarrow J$ is false.

\subsection*{d. Rain or shine, I am studying CS131.}

Rain \textbf{R} 

Shine \textbf{S} 

I am studying CS131 \textbf{I}

$(R \lor S) \rightarrow I$

Since it could either raining or shine right now, $R \lor S$ is always true. I is true. Thus $(R \lor S) \rightarrow I$ is true. 

\subsection*{e. For every $x$: if $x^4>10,000$, then $x>10$.}

$x^4>10,000$ \textbf{A}

$x>10$ \textbf{B}

$A \rightarrow B$

When A is true, B could be false (for $\sqrt[4]{10,000} = \pm 10$, $x <= -10$). Thus $A \rightarrow B$ is false.

\subsection*{f. For every $x$: $x>10$ implies $x^4 > 10,000$.}

$x>10$ \textbf{C}

$x^4>10,000$ \textbf{D}

$C \rightarrow D$

When C is true, D is also true. Thus $C \rightarrow D$ is true.

\subsection*{g. $log x>0$ iff $x>1$.}

$log x>0$ \textbf{L}

$x>1$ \textbf{M}

$L \leftrightarrow M$

When M is true, L is true. When M is false, L is false. There is no way only either M or L is true/false. Thus $L \leftrightarrow M$ is true.

\section*{Question 2}
\subsection*{a. The fact A implies the fact B.}
$A \rightarrow B$
\subsection*{b. The conclusion C follows from the three premises: P, Q, and R.}
$(P \land Q \land R) \rightarrow C$
\subsection*{c. A is sufficient for B.}
$A \rightarrow B$
\subsection*{d. A is necessary for B.}
$B \rightarrow A$
\subsection*{e. A is necessary and sufficient for B.}
$(B \rightarrow A) \land (A \rightarrow B) \equiv A \leftrightarrow B$
\subsection*{f. If I study hard I get good grades.}
I study hard \textbf{S}

I get good grades \textbf{G}

$S \rightarrow G$

\subsection*{g. I get good grades only if I study hard.}
I get good grade \textbf{G}

I study hard \textbf{S}

$G \rightarrow S$.

\subsection*{h. I get good grades if and only if I study hard.}
I get good grade \textbf{G}

I study hard \textbf{S}

$G \leftrightarrow S$.

\section*{Question 3}
\subsection*{2a. The fact A implies the fact B.}
\begin{tabular}{r|l}
    Statement & $A \rightarrow B$ \\
    Contrapositive & $\neg B \rightarrow \neg A$ \\
    & Not B implies not A \\
    Negation & $A \land \neg B$ \\
    & A and not B
\end{tabular}
\subsection*{2f. If I study hard I get good grades.}
I study hard \textbf{S}

I get good grades \textbf{G}

\begin{tabular}{r|l}
    Statement & $S \rightarrow G$ \\
    Contrapositive & $\neg G \rightarrow \neg S$ \\
    & If I don't get good grade, then I didn't study hard. \\
    Negation & $S \land \neg G$ \\
    & I study hard and (still) I don't get good grade
\end{tabular}
\section*{Question 4}
\subsection*{a.}
\begin{tabular}{l|l}
    If Pedro Martinez pitches, the Red Sox win. & $P \rightarrow R$\\
    The Red Sox do not win. & $\neg R$\\ \hline
    Pedro Martinez does not pitch. & $\neg P$
\end{tabular}

This argument is valid as it fit the known valid form of \textit{modus tollens}. 

\subsection*{b.}
\begin{tabular}{l|l}
    If you got a good grade for the course,& \\
    then you did well on the final exam. & $G \rightarrow F$\\
    You did well on the final exam. & $F$\\ \hline
    You got a good grade for the course. & $G$
\end{tabular}

This argument is valid as it fit the known valid form of \textit{modus ponens}. 

\subsection*{c.}
\begin{tabular}{l|l}
    If there is a snow storm, there is no electricity in my house. & $S \rightarrow \neg E$ \\
    The snow is not falling. & $\neg S$ \\ \hline
    My house has electricity. & $E$
\end{tabular}

This argument is invalid as it is the invalid form knows as \textit{fallacy of the inverse}.

\subsection*{d.}
\begin{tabular}{l|l}
    My friend wants to major in CS or biology or chemistry.& $S \lor B \lor C$\\
    She decided that biology and chemistry are not for her. & $\neg B \land \neg C$\\ \hline
    My friend wants to major in CS. & $S$
\end{tabular}

We notice that if we reverse De Morgan's law on the second premise, we have $\neg B \land \neg C \equiv \neg (B \lor C)$. We also notice that we apply associative law and commutative law to have $S \lor B \lor C \equiv (B \lor C) \lor S$.

We have the following argument:

\begin{tabular}{l}
    $(B \lor C) \lor S$\\
    $\neg (B \lor C)$\\ \hline
    $S$
\end{tabular}

This argument is true due to \textit{disjunctive syllogism}.

\subsection*{e.}
\begin{tabular}{l|l}
    If parents go to town, then nanny supervises kids. & $P \rightarrow N$ \\
    If nanny does not supervise kids, then nanny will cook. & $\neg N \rightarrow C$ \\
    Nanny cooked today. & $C$\\ \hline
    Therefore, parents did not go to town today.&$\neg P$.
\end{tabular}

We have the following truth table:

\begin{tabular}{ccc|ccc|c}
	$P$&$N$&$C$&$P \rightarrow N$&$\neg N \rightarrow C$&$C$&$\neg P$ \\ \hline
	\textbf{T}&\textbf{T}&\textbf{T}&\textbf{T}&\textbf{T}&\textbf{T}&\textbf{F}\\
	F&F&F&T&F&F& \\
	T&F&F&F&F&F& \\
	F&T&F&T&T&F& \\
	F&F&T&T&T&T& \\
	T&T&F&T&T&F& \\
	T&F&T&F&T&T& \\
	F&T&T&T&T&T& 
\end{tabular}

The first line has true premises but false conclusion. Thus the argument is invalid.

\subsection*{f.}
\begin{tabular}{l|l}
    If I tell you the time, then we'll start chatting & $T \rightarrow C$  \\
    If we start chatting, & \\
    then I'll have to invite you to my house. & $C \rightarrow H$ \\
    If I invite you to my house, & \\
    then you'll meet my daughter & \\
    and you both will fall in love. & $H \rightarrow (D \land L)$ \\
    If you both fall in love, you'll marry my daughter.&$L \rightarrow M$\\ \hline
    Since I don't want a poor son-in-law without a watch, & \\
    I won't tell you the time. & $\neg M \rightarrow \neg T$
\end{tabular}

First we note that "not having a poor son-in-law without a watch" means that he did not marry the daughter ($\neg M$). Now we need to check if $(H \rightarrow (D \land L)) \land (L \rightarrow M) \equiv H \rightarrow M$. For this, we check if $H \rightarrow (D \land L) \equiv H \rightarrow L$. This only happen if D is always true. 

If D is true, then using transitivity, we can conclude $T \rightarrow M$. Contrapositive of this argument is $\neg M \rightarrow \neg T$. Therefore, the statement is valid (if D is true). Meeting a person doesn't mean that you will fell in love right?


\end{document}

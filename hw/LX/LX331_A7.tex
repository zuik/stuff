\documentclass{article}
\usepackage[utf8]{inputenc}
\usepackage{indentfirst}
\usepackage{amssymb}
\usepackage{fancyhdr}
\usepackage[margin=1.5in]{geometry}

\title{LX331: Assignment 7}
\author{Duy Nguyen}
\date{18 April 2017}

\pagestyle{fancy}
\fancyhf{}
\rhead{Nguyen \thepage}

\begin{document}
\maketitle
\newcommand*{\msim}{\mathord{\sim}}
\newcommand*{\mand}{\mathbin{\&}}

\section{Some subjects are better than others}
\subsection{To \textit{lend} and to \textit{borrow}}
In both (1a) and (1b), we note that both NP \textit{Desmond} and \textit{Mabel} can be assessed under \textsc{Proto-Agent}'s umbrella:

(i) Both NPs are intentional in their participation in the event (the lender have to give the book and the lend has to receive such book). 

(ii) Both NPs are sentient.

(iii) We can argue that the action of transferring the book changes the ownership of the book between one NP to the other NP. 

(iv) There might or might not be a moment involve when we talk about lending or borrowing. There is a movement of ownership of the book, one person need to go and get the book or one need to take the book. 

As we can see, we have a very equivalent amount of \textsc{Proto-Agent} characteristics between \textit{Desmond} and \textit{Mabel} (3 or 4 depend on what actually happens). This explains that either \textit{Desmond} and \textit{Mabel} are eligible to be in the subject NP role. Therefore, there are a pair of VP (\textit{borrow} v. \textit{lend}) to satisfy this need.  

\subsection{But about the \textit{book}?}
We note in our \textit{argument selection principles} that only the "best-fit" \textsc{Proto-Agent} can be in the subject NP's role. Furthermore, we have explore the case where two \textsc{Proto-Agent} are tied in their likeness to the \textsc{Proto-Agent}'s characteristics. Thus we understand that in order to have a predicate that accept some NP in a subject NP's role, that NP has to \textbf{at least equal} in the amount of likeness to the \textsc{Proto-agent} to others argument NPs.

Thus, our job here is to examine NP \textit{book}. We see right away that in the event of lending, a book could not have intent (the book can't lend itself) nor it could have sentient (the book doesn't know it being lend). Thus it failed to be a \textsc{Proto-Agent}. This makes it unable to be in the subject NP.

\section{Translating quantified arguments into First-Order Predicate Logic}

\subsection{A young woman arrived}
\textsc{Young}(a): a is young.

\textsc{Woman}(a): a is female.

\textsc{Arrived}(a): a has arrived.

$x_n$ is a person.

$$ \exists x_1 (\textsc{Young}(x_1) \mand \textsc{Woman} \mand \textsc{Arrived}(x_1)) $$

\subsection{Ida saw something sinister}
\textsc{Sinister}(a): a is sinister

\textsc{See}(a, b): a sees b

$i$ is Ida.

$x_n$ is a thing. 

$$\exists x_2 (\textsc{Sinister}(x_2) \mand \textsc{See}(i, x_2))$$

\subsection{All roads lead to Rome}
\textsc{Road}(a): a is a road.

\textsc{Lead}(a, b): a leads to b.

$r$ is Rome.

$x_n$ is a thing. 

$$ \forall x_3 (\textsc{Road}(x_3) \rightarrow \textsc{Lead}(x_3, r))$$



\subsection{London welcomes all travellers from Spain}
\textsc{Welcome}(a, b): a welcomes b.

\textsc{Traveller}(a, b): a is a traveller from b.

$s$ is Spain.

$l$ is London.

$x_n$ is a person. 


$$\forall x_4 (\textsc{Traveller}(x_4,s) \rightarrow \textsc{Welcome}(l, x_4)) $$


\subsection{There is a castle in Edinburgh}

\textsc{In}(a, b): a is in (at the location of) b.

\textsc{Castle}(a): a is a castle.

$e$ is Edinburgh

$x_n$ is a thing. 

$$\exists x_5 (\textsc{Castle}(x_5) \mand \textsc{In}(x_5, e))$$


\subsection{Someone murdered Clive}
\textsc{Murder}(a, b): a murders b

$c$ is Clive.

$x_n$ is a person. 

$$\exists x_6 (\textsc{Murder}(x_6, c))$$

\subsection{Clive got murdered}

Interestingly, the construction \textit{a got murdered} implied that there was someone intentional murdered a. Therefore, \textit{Clive got murdered} and \textit{Someone murdered Clive} is equivalent.

\textsc{Murder}(a, b): a murders b

$c$ is Clive.

$x_n$ is a person. 

$$\exists x_7 (\textsc{Murder}(x_7, c))$$

\subsection{The boat got sunk}

\textsc{Sink}(a, b): a is sunk by b

$b$ is the boat.

$x_n$ is a thing or a person. 

$$\exists x_8 (\textsc{Sink}(b, x_8))$$


\subsection{The boat sank}
As we have noted above, the construction \textit{Blah got verb-ed} implied there is an actor that caused the event. Therefore, \textit{The boat got sunk} implied that the event of the boat sinking was caused by someone or something. On the other hand, \textit{The boat sank} could be sinking by just accident, which means there might be no one who is an actor in the event.

\textsc{Sink}(a): a sank

$b$ is the boat.

$$\textsc{Sink}(b)$$

\subsection{Nobody saw Charles}
\textsc{See}(a, b): a sees b

$c$ is Charles.

$x_n$ is a person.

$$ \msim \exists x_{10} (\textsc{See}(x_{10}, c))$$


\subsection{Gina or Boris fed every puppy}

\textsc{Feed}(a, b): a feeds b.

$x_n$ is a puppy.

$g$ is Gina.

$b$ is Boris.

$$\forall x_{11} (\textsc{Feed}(g, x_{11}) \lor \textsc{Feed}(b, x_{11}))$$



\section{Quantified arguments and negation}
In order to translate the sentences, we first define our predicate: 

\hspace*{1cm} \textsc{LIKE}(a, b) where \textit{a likes b}. 

We also define that:

\hspace*{1cm} b $=$ Bob, and $x_n$ (where $n \in \mathbb{N}$) $=$ a person. 

(1) \textit{Not everyone likes Bob}:

$$ \msim \forall x_1 (\textsc{Like}(x_1, b))$$

(2) \textit{Bob doesn't like anyone}:

$$ \forall x_1 (\msim \textsc{Like}(b, x_2)) $$

(3) \textit{Bob doesn’t like everyone} can be either:

- There is no one that Bob would like. 

$$ \forall x_3 (\msim \textsc{Like}(b, x_3)) $$


- There are at least someone Bob would like. But for most people, Bob doesn't like them.

$$ \exists x_4 (\msim \textsc{Like}(b, x_4)) $$

(4) \textit{Everyone doesn't like Bob} can be either:

- No one like Bob.

$$ \forall x_5 (\msim \textsc{Like}(x_5, b))$$

- There is someone who likes Bob. But for most people, they don't like Bob.

$$ \exists x_6 (\msim \textsc{Like}(x_4, b)) $$

The ways that (3) and (4) are ambiguous are interesting. We look at the two way that we can translate \textit{every}. 

we can translate it as:

(1) $$\msim \exists x_n (\textsc{Like}(x_n, b)) \equiv \forall x_n (\msim \textsc{Like}(x_n, b)) $$

(2) $$\exists x_n (\msim \textsc{Like}(x_n, b)) $$

This thus gives us an insight into the reason why (3) and (4) were ambiguous.

\end{document}

\documentclass[11pt]{article}
\usepackage{amssymb,amsmath,amsthm,url,graphicx}
\usepackage{fancyhdr}

\def\shownotes{1}   % set 1 for version with author notes
                    % set 0 for no notes



%uncomment to get hyperlinks
%\usepackage{hyperref}

%%%%%%%%%%%%%%%%%%%%%%%%%%%%%%%%%%%%%%%%%%%%%%%%%%%%%%%%%%%%%%
%Some macros (you can ignore everything until "end of macros")

\topmargin 0pt \advance \topmargin by -\headheight \advance
\topmargin by -\headsep

\textheight 8.9in

\oddsidemargin 0pt \evensidemargin \oddsidemargin \marginparwidth
0.5in

\textwidth 6.5in

%%%%%%

\providecommand{\vs}{vs. }
\providecommand{\ie}{\emph{i.e.,} }
\providecommand{\eg}{\emph{e.g.,} }
\providecommand{\cf}{\emph{cf.,} }
\providecommand{\etc}{\emph{etc.} }

\newcommand{\getsr}{\gets_{\mbox{\tiny R}}}
\newcommand{\bits}{\{0,1\}}
\newcommand{\bit}{\{0,1\}}
\newcommand{\Ex}{\mathbb{E}}
\newcommand{\eqdef}{\stackrel{def}{=}}
\newcommand{\To}{\rightarrow}
\newcommand{\e}{\epsilon}
\newcommand{\R}{\mathbb{R}}
\newcommand{\N}{\mathbb{N}}
\newcommand{\Gen}{\mathsf{Gen}}
\newcommand{\Enc}{\mathsf{Enc}}
\newcommand{\Dec}{\mathsf{Dec}}
\newcommand{\Sign}{\mathsf{Sign}}
\newcommand{\Ver}{\mathsf{Ver}}

\providecommand{\mypara}[1]{\smallskip\noindent\emph{#1} }
\providecommand{\myparab}[1]{\smallskip\noindent\textbf{#1} }
\providecommand{\myparasc}[1]{\smallskip\noindent\textsc{#1} }
\providecommand{\para}{\smallskip\noindent}


\newtheorem{theorem}{Theorem}
\theoremstyle{definition}
\newtheorem{ex}{Exercise}
\newtheorem{definition}{Definition}

%%%%%%%  Author Notes %%%%%%%
%
\ifnum\shownotes=1
\newcommand{\authnote}[2]{{ $\ll$\textsf{\footnotesize #1 notes: #2}$\gg$}}
\else
\newcommand{\authnote}[2]{}
\fi
\newcommand{\Snote}[1]{{\authnote{Solution}{#1}}}
\newcommand{\Inote}[1]{{\authnote{Solution}{#1}}}
\newcommand{\Ichanged}[1]{{\authnote{Changed}{#1}}}
%%%%%%%%%%%%%%%%%%%%%%%%%%%%%%%%%

\newcommand{\VAR}{\mathrm{VAR}}



% end of macros
%%%%%%%%%%%%%%%%%%%%%%%%%%%%%%%%%%%%%%%%%%%%%%%%%%%%%%%%%%%%%%


% page counting, header/footer
\usepackage{fancyhdr}
\usepackage{lastpage}
\pagestyle{fancy}
\lhead{\footnotesize \parbox{11cm}{CS558, Boston University, Spring 2014} }
\rhead{Your Name Here:   }
\renewcommand{\headheight}{24pt}

\begin{document}

\title{Homework 1: Threat Modelling\\ Due at 11:59PM on January 27, 2014 as a PDF via websubmit.}
\author{Your name here}
\maketitle

\thispagestyle{fancy}

\myparab{Collaborators: }  Alice, Bob, Eve.

\myparab{Late days for this assignment: } 0

\myparab{Total late days this semester: } 2
 
 
\section*{Answer 1}

Blah blah.

\begin{enumerate}
\item This is a list.
\item This is another list element.
\end{enumerate} 

\noindent\hrulefill

\myparab{Submission policy.} This assignment MUST be submitted as a PDF via websubmit.  It should be \textbf{typed} and \textbf{NOT handwritten} and MUST include the following information:
\begin{enumerate}\itemsep1pt\parsep0pt
\item List of collaborators
\item List of references used (online material, course nodes, textbooks, wikipedia, etc.)
\item Number of late days used on this assignment
\item Total number of late days used thus far in the entire semester
\end{enumerate}
{If any of this information is missing, at least 20\% of the points for the assignment will automatically be deducted from your assignment.  See also discussion on plagiarism on the course syllabus.}

\noindent\hrulefill

\section*{References}

You can enter the references by hand like this:
\begin{enumerate}
\item Sharon Goldberg, Michael Schapira, Pete Hummon, and Jennifer Rexford. How Secure are Secure Interdomain Routing Protocols? SIGCOMM'10, New Delhi, India. August 2010.

\item Kyle Brogle, Sharon Goldberg, and Leonid Reyzin. Sequential Aggregate Signatures with Lazy Verification from Trapdoor Permutations. AISACRYPT 2012. Beijing, China. December 2012.
    
\item Wikipedia. Kerckhoff's Principle. Available at \url{http://en.wikipedia.org/wiki/Kerckhoffs's_principle}.
\end{enumerate}


\end{document} 
\documentclass{article}
\usepackage[utf8]{inputenc}
\usepackage{listings}
\usepackage{color}
\usepackage{indentfirst}
\lstdefinestyle{mystyle}{
  basicstyle=\footnotesize,
  breakatwhitespace=false,         
  breaklines=true,                 
  captionpos=b,                    
  keepspaces=true,                 
  numbers=left,                    
  numbersep=5pt,                  
  showspaces=false,                
  showstringspaces=false,
  showtabs=false,                  
  tabsize=2
}
\lstset{style=mystyle}

\title{CS210: Problem set \#1}
\author{Duy Nguyen}
\date{10 February 2017}



\begin{document}

\maketitle

\section*{1.}
\begin{lstlisting}[language=c]
unsigned replace_byte (unsigned x, int i, unsigned char b){
    // Bit mask x to create a hole so that we can insert b
    return ~((~(x | (0xFF << i*8)) | 
                (b << i*8)) ^ 
            (0xFF << i*8));
}
\end{lstlisting}
\section*{2.}
\begin{lstlisting}[language=c]
// A. Any bit of x equals 1.
int bit_equals_1(int x){
    return x || 0;
}
// B. Any bit of x equals 0.
int bit_equals_0(int x){
    return ((x ^ INT_MAX)) && 1;
}
// C. Any bit in the least significant byte of x equals 1.
int lsb_equals_1(int x){
    // Destroy all bit except the last then compare it with 0
    return ((x << 24) >> 24) || 0;
}
// D. Any bit in the most significant byte of x equals 0.
int msb_equals_0(int x){
    // Similar to C, but on the other direction
    return (((x >> 24) ^ INT_MAX) ^ INT_MIN) || 0;
}
\end{lstlisting}
\section*{3.}
\begin{lstlisting}[language=c]
/* Return 1 when any odd bit of x equals 1; 0 otherwise.
Assume w=32. */
int any_odd_one(unsigned x){
// Bit mask all the odd bit
    return (0xAAAAAAAA & x) && 1;
}
\end{lstlisting}
\section*{4.}
A. The code has the right idea but didn't care for the signed-ness of the bytes extracted. Therefore, for the positive number, it will return the correct answer, it will not return the correct answer for the negative number though.

B. 
\begin{lstlisting}[language=c]
int xbyte(packed_t word, int bytenum){
    bytenum = bytenum << 3;
    //Extract the byte like the "predecessor"
    int byte_extracted = (word >> bytenum) & 0xFF; 
    // Go get the correct sign
    return ((byte_extracted - (1 << 8)) << 24) >> 24;
}
\end{lstlisting}
\section*{5.}
\begin{lstlisting}[language=c]
int tsub_ok(int x, int y){
    int sb_a = (unsigned)x >> 31;
    int sb_b = (unsigned)y >> 31;
    int sb_c = (unsigned)(x-y) >> 31;
    // We compare the sign and see if there is a weird sign change happened.
    return ((sb_a ^ sb_b) & ~(sb_b ^ sb_c)) ^ 1;
}
\end{lstlisting}
\section*{6}
a. $(010101)_2 = 21_{10}$ 

b. $(011100)_2 = 1C_{16}$

c. $54.125_{10} = (110110.001)_2$

d. $(122.3)_4 = 26.75_{10} = 11010.11_2 = 1A.C_{16} $

\section*{7.}
    (a) 1 1110 + 1 1101 = 11 0011. Not overflowed. The result was correct.
    
    (b) 1 0111 + 1 0111 = 10 1110. Overflowed: Two negative number add to a positive number.
    
    (c) 1 1111 + 0 1011 = 10 1010. Not overflowed. The result was correct.

\section*{8.}
\begin{tabular}{l|c|r}
& Size & Bit value \\ \hline
char c = 35; & 1 byte & 0010 0011  \\
char d = 'G';& 1 byte & 0100 0111 \\
int x = -42; & 4 byte & 1111 1111 1111 1111 1111 1111 1101 0110
\end{tabular}
\section*{9.}
\begin{lstlisting}[language=c]
#include <stdio.h>
main() {
    char n[4];
    int x;
    printf("Enter a 3-digit non-negative number: ");
    scanf("%s", n);
    x = ((n[0] & 0xF) << 8) | ((n[1] & 0xF) << 4) | (n[2] & 0xF);
    printf("The number is %x \n", x);
    return 0;
}
\end{lstlisting}
\end{document}

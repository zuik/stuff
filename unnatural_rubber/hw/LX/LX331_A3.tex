\documentclass{article}
\usepackage[utf8]{inputenc}
\usepackage{indentfirst}
\usepackage{amssymb}
\usepackage{fancyhdr}

\title{LX331: Assignment 3}
\author{Duy Nguyen}
\date{17 February 2017}

\pagestyle{fancy}
\fancyhf{}
\rhead{Nguyen \thepage}

\begin{document}

\maketitle

\section{The syntax and semantics of Propositional Logic}
\subsection*{A. B.}
(1) $\sim s \& r$ : Stuart is not in the kitchen and Fred left.

\begin{tabular}{cc|ccccc}
r & s & $\sim s$ \& $\sim s \& r$ \\
\hline 
T & T & F & F \\
T & F & T & T \\
F & T & F & F \\
F & F & T & F \\
\end{tabular}

(2) $\sim (p \lor q)$ : It is not true that Mary or Sue is at home.

\begin{tabular}{cc|ccccc}
p & q & $p \lor q$ \& $\sim (p \lor q)$ \\
\hline 
T & T & T & F \\
T & F & T & F \\
F & T & T & F \\
F & F & F & T \\
\end{tabular}

(3) $\sim (p \& s) \rightarrow q$ : If it is not true that Marry is at home and Stuart is in the kitchen, then Sue is at home.

\begin{tabular}{ccc|ccccc}
p & q & s & $p \& s$ & $\sim (p \& s)$ & $\sim (p \& s) \rightarrow q$ \\
\hline 
T &T & T & T & F & T \\
T &T & F & F & T & T\\
T &F & T & T & F & T\\
T &F & F & F & T & F\\
F &T & T & F & T & T\\
F &T & F & F & T & T\\
F &F & T & F & T & F\\
F &F & F & F & T & F\\
\end{tabular}

(4) $\sim q \& ((s \lor \sim r) \rightarrow \sim s)$ : Sue isn't at home and if Stuart is in the kitchen or Fred didn't left then Stuart is not in the kitchen. 

\begin{tabular}{ccc|ccccc}
p &r & s & $s \lor \sim r$ & $(s \lor \sim r) \rightarrow \sim s$ & $\sim q \& ((s \lor \sim r) \rightarrow \sim s)$ \\
\hline 
T &T & T & T & F & F \\
T &T & F & F & T & F\\
T &F & T & T & F & F\\
T &F & F & T & T & F\\
F &T & T & T & F & F\\
F &T & F & F & T & T\\
F &F & T & T & F & F\\
F &F & F & T & T & T\\
\end{tabular}

\subsection*{C. D.}
(5) Stuart is not in the kitchen and Fred didn’t leave. 

$\sim s \& \sim r$

We can rewrite this logic formula as $\sim (s \lor r)$ due to DeMorgan's law.

\begin{tabular}{cc|c}
s & r & $\sim s \& \sim r \equiv \sim (s \lor r)$ \\
\hline 
T & T & F  \\
T & F & F \\
F & T & F  \\
F & F & T  \\
\end{tabular}

(6) If Sue is at home or Fred didn’t leave, then Stuart is not in the kitchen.

$(q \lor \sim r) \rightarrow \sim s$

\begin{tabular}{ccc|cc}
p & r & s & $(q \lor \sim r)$ & $(q \lor \sim r) \rightarrow \sim s$ \\
\hline 
T & T & T & T & F  \\
T & T & F & F & T \\
T & F & T & T & F \\
T & F & F & F & T \\
F & T & T & F & T \\
F & T & F & F & T \\
F & F & T & F & F \\
F & F & F & F & T \\
\end{tabular}


(7) It's not the case that Mary and Sue are both at home.

$\sim (p \& q)$

\begin{tabular}{cc|c}
p & q & $\sim (p \& q)$ \\
\hline 
T & T & F  \\
T & F & T \\
F & T & T  \\
F & F & T  \\
\end{tabular}

(8) Neither Mary nor Sue is at home.

$\sim (p \lor q)$

\begin{tabular}{cc|c}
p & q & $\sim (p \lor q)$ \\
\hline 
T & T & F  \\
T & F & F \\
F & T & F  \\
F & F & T  \\
\end{tabular}

\section{Representing semantic ambiguity in Propositional Logic}

(9) I didn't talk to Fred and Barney

We can interpret two understanding from (9).

(a) I did not talk to Fred and Barney at the same time. 

$\sim (p \& q)$

(b) I did not talk to Fred, and I did not talked to Barney

$\sim p \& \sim q$

If we look into the truth table, we can see that the two interpretations of the English sentence is not equivalent. 

\begin{tabular}{cc|cc}
p & q & $\sim (p \& q)$ & $\sim p \& \sim q$\\
\hline 
T & T & F & F  \\
T & F & T & F\\
F & T & T & F\\
F & F & T & T \\
\end{tabular}

\section{A new logical connective}

I can't type the connective arrow so $->$ is a substitution. 

\begin{tabular}{cc|cl}
A & B & A $->$ B & Comment\\
\hline 
T & T & F & It's the last week so there can't be homework  \\
T & F & T & It's not the last week so there have to be homework\\
F & T & T & It's the last week, so it is true that there isn't homework\\
F & F & F & It's not the last week, so if there is no homework it is false. \\
\end{tabular}

\section{Logical relations between sentences}

\subsection*{A.}
So we have that $p \rightarrow q \equiv \sim p \lor q \equiv \sim (p \& \sim q)$. So $p \rightarrow q$ and $\sim (p \& \sim q)$ should be equivalent. Let's check with the truth table.

\begin{tabular}{cc|cc}
p & q & $p \rightarrow q$ & $\sim (p \& \sim q)$ \\ \hline 
T & T & T & T \\
T & F & F & F \\
F & T & T & T \\
F & F & T & T \\
\end{tabular}

\subsection*{B.}
Logical compatible means that there are rows in that both formulae is both true or both false.

\begin{tabular}{cc|cc}
r & s & $\sim r \& \sim s$ & $\sim (r \lor \sim s)$ \\ \hline 
\textbf{T} & \textbf{T} & \textbf{F} & \textbf{F} \\
\textbf{T} & \textbf{F} & \textbf{F} & \textbf{F} \\
F & T & F & T \\
F & F & T & F \\
\end{tabular}

As we can see, the first two row are show that these two formulae are compatible.

\subsection*{C.}
One formula entails the other when the truth of the first "forces" the truth of the second. The table for two formulae is as follow:

\begin{tabular}{cc|cc}
p & s & $ \sim (p \& s) $ & $ \sim (p \lor s) $ \\ \hline 
T & T & F & F \\
T & F & T & F \\
F & T & T & F \\
\textbf{F} & \textbf{F} & \textbf{T} & \textbf{T} \\
\end{tabular}

From the truth table, we can see that $ \sim (p \& s) $ does not entails $ \sim (p \lor s) $. However, the last line show us that $ \sim (p \lor s) $ entails $ \sim (p \& s) $.


\end{document}

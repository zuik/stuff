\documentclass{article}
\usepackage[utf8]{inputenc}
\usepackage{indentfirst}

\title{CS131 B1: HW 3}
\author{Duy Nguyen}
\date{2 October 2016}

\begin{document}

\maketitle

\section*{Question 1}
\subsection*{a.}
\begin{tabular}{l|l}
    The program has bugs or works correctly. & $B \lor C$ \\
    The program is bug-free or needs debugging. & $\neg B \lor N$ \\ \hline
    the program works correctly or needs debugging.&$C \lor N$.
\end{tabular}

This argument is valid since it follows the valid argument form of \textit{resolution}. 
\subsection*{b.}
\begin{tabular}{l|l}
    The error is in function A or B. & $A \lor B$ \\ \hline
    I need to debug function A & $A$
\end{tabular}

We construct a truth table:

\begin{tabular}{cc|cc}
    $A$&$B$&$A \lor B$&$A$ \\ \hline
    T&T&T&T\\
    T&F&T&T\\
    \textbf{F}&\textbf{T}&\textbf{T}&\textbf{F}\\
    F&F&F&F
\end{tabular}

We note that there is one case where a true premise leads to a false conclusion. Thus this argument is invalid. 

\subsection*{c.}
\begin{tabular}{l}
    All integers are rational numbers. \\
    All rational numbers can be represented as $p/q$, \\
    where
p and q are integers. \\ 
    There are no such p and q which satisfy $x=p/q$. \\ \hline
    x is irrational and x is not an integer.
\end{tabular}

 
 We have: ``all integers are rational numbers" and ``all rational numbers can be represent as $p/q$" are both true since this is the definition of rational number. So if the third premise ``there are no such p and q which satisfy $x=p/q$" is true, then x is not an rational number. And since integers are subset of rational numbers, x is not an integer. Therefore, this argument is valid.
 
 \subsection*{d.}
 \begin{tabular}{l}
     All the planets of the solar system orbit the Sun. \\
     Pluto orbits the Sun. \\ \hline
     Pluto is a planet of
the solar system.
 \end{tabular}

Both of the premises are true. However, the conclusion is false. Therefore, this is not a valid argument.

\section*{Question 2}
\subsection*{a. Every CS student knows Python and Java.}
$\forall x (C(x) \rightarrow (P(x) \land J(x)))$

\begin{tabular}{ll}
    Where: & \textbf{x} are all people. \\
    & \textbf{C(x)} are those who are a CS student. \\
    & \textbf{P(x)} is a person who knows Python.\\
    & \textbf{J(x)} is a person who knows Java.
\end{tabular}

\subsection*{b. Some CS students know C++ or C\#.}
$\exists x (C(x) \land (P(x) \lor S(x))$

\begin{tabular}{ll}
    Where: & \textbf{x} are all people. \\
    & \textbf{C(x)} are those who are a CS student. \\
    & \textbf{P(x)} is a person who knows C++.\\
    & \textbf{S(x)} is a person who knows C\#.
\end{tabular}

\subsection*{c. For every successful person there is someone even more successful, for every unhappy person there is someone more unhappy.}

$\forall x \exists y P(x,y)$

\begin{tabular}{ll}
    Where: & \textbf{x} is a successful/unhappy person. \\
    & \textbf{P(x,y)} is where y is more successful/unhappy than x 
\end{tabular}

\subsection*{d. If y=f(x) is a function, then for every x there exists only one y.}
$ y = f(x) \rightarrow \forall x \exists y $

\section*{Question 3}
\subsection*{2a.}
$\neg \forall x (C(x) \rightarrow (P(x) \land J(x)))$

$\exists x \neg (C(x) \rightarrow (P(x) \land J(x)))$

$\exists x (C(x) \land \neg (P(x) \land J(x)))$

$\exists x (C(x) \land (\neg P(x) \lor \neg J(x)))$

There exists a person who is a CS student and does not know Python nor Java.

\subsection*{2b.}
$\neg \exists x (C(x) \land (P(x) \lor S(x))$

$\forall x \neg (C(x) \land (P(x) \lor S(x))$

$\forall x (\neg C(x) \lor \neg (P(x) \lor S(x))$

$\forall x (\neg C(x) \lor (\neg P(x) \land \neg S(x))$

For all people, there is not a CS student or there is not a person who know both C++ and C\#.

\end{document}

\documentclass{article}
\usepackage[utf8]{inputenc}
\usepackage{indentfirst}

\title{CS131 B1: HW 4}
\author{Duy Nguyen}
\date{10 October 2016}

\begin{document}

\maketitle

\section*{Question 1}
\subsection*{a.$\forall x \forall y P(x,y)$}
When $ x \neq y $ the statement is \textbf{false}.
\subsection*{b.$\forall x \exists y P(x,y)$}
When $ x = 0 $ there is no $y$ where $ \frac{0}{y} = 1$. \textbf{False}.
\subsection*{c.$\exists x \forall y P(x,y)$}
There is not an $x$ where it could satisfy the condition. Alternatively, $y=0$ would blow up the statement for every $x$. \textbf{False}.
\subsection*{d.$\exists x \exists y P(x,y)$}
When $x = y$, and $y \neq 0$, the statement is true\textbf{true}.

\section*{Question 2}
\subsection*{a. $\exists x \forall y (x+y=y)$}
There exists an $x$ such that for every $y$, $x + y = y$.

This is \textbf{true} when $ x = 0$.

\subsection*{b. $\forall x \forall y ((x \geq 0) \land (y < 0) \rightarrow (x - y > 0))$}

For all $x$ greater or equal to zero, and $y$ which is a negative number, $x - y > 0$.

Since $x > y$, the equation $x - y$ is greater than 0. \textbf{True}.

\subsection*{c. $\forall x \forall y ((x \neq 0) \land (y \neq 0) \leftrightarrow (xy \neq 0))$}

For all $x$ and $y$ beside zero, their product is not zero. 

Since there is no zero being multiply, the result could not be zero. Thus the statement is \textbf{true}.

\section*{Question 3}
\subsection*{a. Any person who is a female and a parent, is a mother of someone.}

\begin{tabular}{rl}
    Define: & $x$ is a person \\
            &    $y$ is another person \\
            &    $F(x)$ is a female person \\
            &    $P(x)$ is a parent \\
            &    $M(x,y)$ said x is the mother of y
\end{tabular}


Then we have:

$$\forall x \exists y (F(x) \land P(x) \rightarrow M(x,y))$$

\subsection*{b. There are people who are friends with everyone, people who have some friends, and
people who have no friends.}

\begin{tabular}{rl}
    Define: & $a, b, c, x$ are people \\
            & $F(x, y)$ said that x is friend with y 
\end{tabular}

Then we have:
$$ (\exists a \forall x F(x, y)) \land 
(\exists b \exists x F(x, y) \land 
(\exists c \forall x \neg F(x,y))$$

\subsection*{c. For every real number $C>0$, there exists $\epsilon > 0$, such that $f(x) > C$ whenever $0 < x < \epsilon$.}

\begin{tabular}{rl}
    Define: & $C$ is a real number $> 0$ \\
    & $\epsilon$ is a small real number $>0$\\
    & $x$ is a real number \\
    & $f(x)$ is a function on x
\end{tabular}

Then we have that:
$$ \forall C \exists \epsilon (0 < x < \epsilon) \rightarrow (f(x) > C)$$ 

\subsection*{d. Negation of the previous statement in logical form and in English.}
$$ \forall C \exists \epsilon (0 < x < \epsilon) \rightarrow (f(x) > C)$$ 

Negated:

$$ \exists C \forall \epsilon \neg ((0 < x < \epsilon) \rightarrow (f(x) > C))$$ 
$$ \exists C \forall \epsilon ((0 < x < \epsilon) \land \neg (f(x) > C))$$
$$ \exists C \forall \epsilon ((0 < x < \epsilon) \land (f(x) \leq C))$$

There exists $C$, a real number greater than $0$ where for all $\epsilon$: $x$ is between $0$ and $\epsilon$, and  $f(x)$ is less or equal to $C$
\end{document}

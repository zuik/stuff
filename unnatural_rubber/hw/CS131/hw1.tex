\documentclass{article}
\usepackage[utf8]{inputenc}
\usepackage{indentfirst}

\title{CS131 B1: HW 1}
\author{Duy Nguyen}
\date{13 September 2016}

\begin{document}

\maketitle

\section*{Question 1}
In the truth table for $P \land Q$, when P is true and Q is true, the result is true. However, this case should be false if it is a truth table for $P \oplus Q$.

\begin{table}[h]
\centering
\caption{Question 1b}
\begin{tabular}{cc|cc}
P & Q & $P \oplus Q$ & $(P \land \neg Q) \lor (Q \land \neg P)$ \\ \hline
T & T & F          & F                                    \\
T & F & T          & T                                    \\
F & T & T          & T                                    \\
F & F & F          & F                                   
\end{tabular}

\end{table}

Therefore, $P \oplus Q \equiv (P \land \neg Q) \lor (Q \land \neg P)$.

\section*{Question 2}
\begin{table}[h]
\centering
\caption{2a. First De Morgan's law}
\begin{tabular}{cc|cc}
P & Q & $\neg (P \land Q)$ & $\neg P \lor \neg Q$ \\ \hline
T & T & F          & F                                    \\
T & F & T          & T                                    \\
F & T & T          & T                                    \\
F & F & T          & T                                   
\end{tabular}

\end{table}

\begin{table}[h]
\centering
\caption{2b. First Domination Law}
\begin{tabular}{cc|cc}
P & T & $P \lor T$ & $T$ \\ \hline
T & T & T          & T                                    \\
F & T & T          & T                                    
\end{tabular}
\end{table}

\begin{table}[h]
\centering
\caption{2c. First Negation Law}
\begin{tabular}{cc|cc}
P & $\neg P$& $P \lor \neg P$ & $T$ \\ \hline
T & F & T          & T                                    \\
F & T & T          & T                                    
\end{tabular}
\end{table}

\begin{table}[h]
\centering
\caption{2d. First Absorption Law}
\begin{tabular}{cc|cc}
P & Q & $P \lor (P \land Q)$ & $P$ \\ \hline
T & T & T          & T                                    \\
T & F & T          & T                                    \\
F & T & F          & F                                    \\
F & F & F          & F                                   
\end{tabular}
\end{table}
\section*{Question 3}
\begin{tabular}{ll}
$\neg (\neg P \land Q) \lor (P \land \neg R)$ & \\
$(P \lor \neg Q) \lor (P \land \neg R)$ & (De Morgan's law) \\
$P \lor (P \land \neg R) \lor  \neg Q$ & (Commutative) \\
$(P \lor (P \land \neg R)) \lor  \neg Q$ & (Associative) \\
$P \lor \neg Q$ & (Absoption) \\
\end{tabular}

Therefore, $\neg (\neg P \land Q) \lor (P \land \neg R) \equiv P \lor \neg Q$. 

\section*{Question 4}
\subsection*{4a. I can program using Python or Java, but not C\#.}

I can program using Python is \textbf{P}

I can program using Java is \textbf{J}

I can program using C\# is \textbf{C}

Then our statement is $(P \lor J) \land \neg C$.

\subsection*{4b. The number of a leap year is divisible by 4 but not divisible by 100, unless it is divisible by 400.}

Leap year divisible by 4 is \textbf{A}

Leap year divisible by 100 is \textbf{B}

Leap year divisible by 400 is \textbf{C}

Then our statement is $A \land \neg B \land C$.

\subsection*{4c. If I study hard, then I get good grades.}

I study hard is \textbf{H}

I get good grade is \textbf{G}

Then our statement is $H \rightarrow G$.

\subsection*{4d. Pluto is called a planet if and only if the sky bodies located in Kuiper Belt are called planets.}

Pluto is called a planet is \textbf{P}

Sky bodies located in Kuiper Belt are called planets is \textbf{K}

Then our statement is $P \leftrightarrow K$.

\end{document}
